% =========================================================================
% CHAPTER 6
% =========================================================================

\chapter{Diskussion}
\label{K6}

\section{Fazit}
Der folgende Abschnitt diskutiert die Vor- und Nachteile von
on demand verteilten JobScripts,
der tiefen Position der Middleware im Schichtenmodel,
dem asyncronen Job API,
dem \JobTree{} Konzept
und der \UI{} Anbindung.


\begin{multicols}{2}
\subsection{Code on demand}
Die Implementierung hat gezeigt, dass die gewählten Technologien gut geeignet sind, um on demand verteilte Scripts von der Middleware ausführen zu lassen.
Der on demand verteilte Code eröffnet jedoch Angriffsmöglichkeiten.
Ein möglicher Lösungsansatz wären abgesicherte Sandboxes am Server und den Workern, wobei bedacht werden muss, dass das Starten von Prozessen ein gewünschtes Feature ist.
Die Verwendung von Secure Websockets, einer Benutzeranmeldung am Client und eine Assoziation der Websocket Verbindung mit einem Benutzer am Server wäre ebenso denkbar.
Dabei ist jedoch zu beachten, dass für jeden Benutzer des Webservers auch ein Benutzer im Betriebssystem angelegt werden muss.


\subsection{Jobs und Promises}
Das asynchrone API ist geeignet um Roundtrip arme \jobScript s zu schreiben, allerdings auch komplizierter als ein synchrones.
Die Ahnlichkeit zu \JavaScript{} Promises könnte zu einer Kompartiblität ausgebaut werden.
Jeder Job kann mit einem Timeout versehen werden.
Netzwerkprogrammierung erfordert häufig die Berücksichtigung von Timeouts, meist im Zusammenhang mit asynchronen Vorgängen.
Die Vereinigung dieser beiden Features in einer Klasse ermöglicht eine kompakte Implementierung der Requirments mit 1316 Zeilen JavaScript.
Timeouts finden auch bei lokalen asynchronen Operationen Anwendung.
Bei Promises kann man zwar Hardware Fehler vernachlässigen, aber Implementierungsfehler, wie zum Beispiel das fehlende Schließen einer Promise wird nicht erkannt.
Eine solche fehlerhafte Implementierung wird bei Jobs ein Timeout auslösen, was die Fehlersuche erleichtert.

\subsection{Low-level}
Das Middleware API ist auf einer tiefen Ebene im Schichtenmodell angesiedelt.
Daraus resultiert, dass Scheduler Algorithmen (\ref{schedulerNeeded}), Vermeidung von Überlastungen (\ref{overload}) und Workflow Logiken im \ApplicationLayer{} realisiert werden müssen.
Um diesen Umstand zu verbessern, könnten weitere Layer darüber geschaffen werden,
zum Beispiel nach dem Vorbild von \MapReduce{} oder \ApacheSpark{}.

\subsection{\JobTree s}
Mit der Hilfe der Job.delegate Funktion können Abhängigkeiten zwischen den Jobs deklariert werden.
Die Middleware kann so zur Laufzeit einen \JobTree{} aufbauen, der zu Debugging und Usability Zwecken verwendet werden kann.
Zum Beispiel ist im Fehlerfall daraus ersichtlich, auf welchem Gerät der Fehler stattgefunden hat.
Auch Ablaufvisualisierungen und Performance-Analysen sind damit möglich.

\subsection{\UI{} Anbindung}
Die onUpdate und onReturn Events der \RootJob s können für die Anbindung des \UI{} verwendet werden.
Die Anbindung ist sehr einfach, da die beiden Events alle durch das Netzwerk veruhrsachten \UI{} Updates abdecken (siehe Abbildung \ref{seq}).
Das gilt für das Anzeigen von Ergebnissen genauso wie für die Anzeige des Progress, Enablen/Disable Events und Fehlermeldungen - egal ob lokal oder remote.
Cancel Implementierungen sind vereinfacht auf das triviale Auslösen eines Abbruchs mit einem Funktionsaufruf.
Fehlermeldungen können ebenfalls im \UI{} dargestellt werden, indem im onReturn Event auf die Zustände der SubJobs geachtet wird.
\end{multicols}


%\item Der Anwender ist in der Lage, beliebige Verteilungsstrategien mit Hilfe von \jobScript s zu definieren, ohne einen Deployment Process auszuführen.
%Das Skript wird während der Ausführung an die entsprechenden Geräte verteilt - dies könnte auch als Deployment zur Runtime gesehen werden - jedenfalls geschieht es automatisch.
%\section{\RootJob{} Events und \UI{}}


\clearpage
%\section{Konklusion}
%\chapter{Fazit}
\section{Zukünftige Arbeiten}

Es wurden \jobScript s für zwei Network Overlays implementiert, Client Server und \hcsno{} mit einer Worker Ebene.
Für diese Overlays wurden jeweils mehrere Test Scripts mit geringem Zeitaufwand  implementiert.
Das Konzept könnte Anwendung bei der Implementierung von Prototypen finden.
Die Network Overlay Struktur ist vorerst aber noch statisch zur Runtime und eine Änderung würde die Anpassung von Kernkomponenten erfodern.
Eine Erweiterung um ein \ptp{} Network Overlay würde einen interessanten Raum für Experimente schaffen.


Kapitel \ref{K7} zeigt die Skalierbarkeit verschiedener Algorithmen verteilt auf ein \hcsno{} mit einer Worker Ebene.
Es ist ein grundlegendes Problem, dass ein Server nur eine begrenzte Anzahl von Workern bedienen kann.
Die \JavaScript{} Implementierung kann im worst Case Vier Worker pro Server bedienen (siehe \ref{M3}).
Eine sehr große Anzahl von Workern ist nur möglich, wenn nicht alle Worker mit demselben Server verbunden sind.
Ein hierarchisches Netzwerk mit mehreren Ebenen könnte für das Starten und Stoppen von Prozessen auf Worker eine Laufzeit  $T(n) = \mathcal{O}(log_{w}(n))$ erreichen, wobei $n$ die Gesamtanzahl der Worker und Server ist, und $w$ die Anzahl der Worker pro Server.
Auch die \ProgressAnzeige{} konnte dabei erhalten bleiben.


Derzeit enthält der Webclient mehrere Visualisierungen:
den Workflow Graph zu jedem Run eines Scripts, einen Gant Chart, der die Laufzeiten aller Jobs zeigt, sowie eine vereinfachte Darstellung des Netzwerks.
Sie alle visualisieren dasselbe Model, den \JobTree{}.
Für dieses Model könnten weitere Visualisierungen geschaffen werden, die bei Analysen helfen würden.
Der \JobTree{} könnte auch verwendet werden um mit Hilfe von Klassifikationsverfahren Fehler und Abweichungen zu erkennen.
Abbildung \ref{hist-workerBacc0} zeigt, dass einige WorkerJobs praktisch sofort termininierten, was auf ein Fehlverhalten hinweist.


Das Job API kann noch weiter vereinfacht werden.
Eine Kompatibilität zu JavaScript Promises würde bedeuten, dass Anwender die mit diesem schon vertraut sind, das Job API schnell erlernen würden.
Eine Trennung des Jobs in zwei Komponenten, Future und Promise nach \cite{baker1977incremental}, scheint aber auch sinnvoll.



%\item security
%\item onCall für ander programmirsprachen / onCall dependanxls
