% =========================================================================
% CHAPTER 3
% =========================================================================

\chapter{Anforderungen und verwendete Technologien}
\label{K3}
In diesem Kapitel werden zunächst die Anforderungen an die Middleware zusammengefasst und analysiert.
Danach die verwendeten Technologien und Konzepte aufgelistet und begründet, warum die Entscheidung auf sie gefallen ist.



\section{Anforderungen}

\BCL
  \item Es soll eine Middleware geschaffen werden, die auf einem möglichst großen Spektrum von Betriebssystemen und Geräten lauffähig ist.

  \item \ApplicationLayer{} sollten als Scripts realisiert werden, die von der Middleware ausgeführt werden.
  Diese Scripts werden im folgenden \jobScript{} genannt.

  \item Der Deployment Aufwand soll minimiert werden.
  Wie bei Apache Spark sollte der User nur ein Script schreiben müssen, welches das Ausführungsverhalten des gesamten Systems definiert, also auch den auf anderen Geräten ausgeführten Code enthält.
  Die Teile des Scripts welche auf anderen Geräten ausgeführt werden, sollten on demand auf diese verteilt werden.

  \item  \jobScript s sollten frei von Synchronisierungsmechanismen wie Semaphoren sein, da ansonsten zu viel technisches Know-how für die Implementierung von \jobScript s notwendig ist.

  \item JobsScripts sollen es ermöglichen bei Funktionsaufrufen ein Gerät zu spezifizieren, auf dem die Funktion ausgeführt wird.
  Da bei \remoteJob s das Gerät explizit angegeben werden muss, muss innerhalb der Scripting Sandbox zumindest Zugriff auf die verbundenen,
  oder alle im Netzwerk zur Verfügung stehenden Geräte gegeben sein.
  Somit kann der Anwender eigene Scheduling Strategien implementieren.
  Pro Rechner muss CPU Auslastung, verfügbarer Speicher und verfügbare Interpreter abgelesen werden können.
  Im Folgenden wird diese Datenstruktur \netInfo{} genannt, siehe \ref{netInfo}.

  \item Argumente und Returnwert oder Exceptions sollen von der Middleware transportiert werden.

  \item Auch das Starten von Remote Prozessen sollte auf dieses Art möglich sein.
  Solche Remote Calls werden im folgenden RemoteJobs genannt.
  Es wird angenommen, dass \remoteJob s mit einer sehr langen Ausführungszeit zum Einsatz kommen, deshalb müssen diese abbrechbar sein.
  Außerdem muss es möglich sein, den Progress von \remoteJob s an der auftraggebenden Stelle abzulesen.
\ECL





\clearpage
\section{Verwendete Technologien}

\subsection{ECMA Script aka JavaScript}
Ein \jobScript{} enthält den Code aller beteiligten Geräte.
Funktion hohrere Ordnung enthalten Funktionen in der Argumentliste.
Bei Remote Aufrufen müssen die Argumente an die Gegenstelle gesendet werden, bei Porgrammiersprachen wie C++ ist es schwierig eine Serialisierbare Repräsentation einer Funktion zu erhalten.
Die Implementierung eines eigenen Compilers oder Präprozessors soll umgangen werden.
Programmiersprachen mit Reflection würden dies ermöglichen, allerdings mit erheblichem Aufwand.
\JavaScript{} macht es sehr einfach Stringrepräsentationen einer Funktion zu erhalten.

Javascript Interpreter sind auf nahezu allen Betriebssystemen vorhanden.
Auch mobile Endgeräte sind meist mit Browsern ausgestattet, die Javascript unterstützen.
Dadurch ist JavaScript eine der am häufigsten zur Verfügung stehenden Laufzeitumgebungen.
JavaScript unterstützt kein Multithreading - auf Servern werden Prozesse anstatt Threads verwendet \cite{nodecite}.
Webworker unterstützen kein Shared Memory, und werden im folgenden nicht als Threads betrachtet.




\subsection{Websockets}
Für die Synchronisierung der \netInfo{} Datenstruktur muss bei Änderungen der Geräteeigenschaften eine Message an den Server und von diesem weiter an die Clients gesendet werden.
Vor allem das Weiterleiten an die Clients kann mit Websockets effizienter, das heißt ohne Polling, realisiert werden.
\remoteJob s müssen dem auftraggebenden Gerät Progress Updates zukommen lassen.
Auch hier kann Polling durch die Verwendung von Websockets vermieden werden.




\subsection{\node{} und \ActiveObjectPattern{}}
\JavaScript{} in Browsern und \node{} verwenden ein Threading Model, das einem \ActiveObjectPattern{} entspricht, wobei ein ganzer \node{} Prozess sowie ein Web Worker als \ActiveObjectPattern{} gesehen werden kann.
Darauf aufbauend kann eine MOM sehr einfach implementiert werden.
Um die Terminierung, das Abbrechen und Timeouts von \remoteJob s zu realisieren, muss eine \StateMachine{} auf Client- sowie Serverseite implementiert werden (siehe Kapitel \ref{K4}).
Die Transitions dieser State Machine werden durch Netzwerk Messages, User Inputs und Timer Events ausgelöst.

Da Webbrowser und \node{} kein Multithreading mit Shared Memory unterstützen, ist ein Asynchronous API unumgänglich.
Ein synchrones API würde entweder eine hohe Anzahl von Prozessen oder Web Worker benötigen, beziehungsweise starke Latenz Einbußen mit sich bringen.

%\subsection{MOM}
