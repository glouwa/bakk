% =========================================================================
% CHAPTER 2
% =========================================================================

\chapter{Ähnliche Arbeiten und Technologien}
\label{K2}
\section{Konzepte}

\subsection{Message-Oriented Middleware vs RPC}
Middleware Systeme können nach mehreren Gesichtspunkten klassifiziert werden.
Im Folgenden wird die Klassifizierung nach \cite{bishop2003survey} verwendet und auf die Klassen \sys{Procedure Oriented}, \sys{Object Oriented}, und \sys{Message-Oriented Middleware} eingegangen, da sie unter Berücksichtigung der Requirements mögliche Lösungsansätze sind.

\sys{Procedure Oriented} und \sys{Object Oriented} Systeme sind häufig als Request-Response  Protokolle implementiert, wie zum Beispiel Java RPC oder CORBA.
Einen Remote Call abzubrechen ist in diesem Kontext nur möglich, indem man die Ausführung an die TCP Connection bindet, das hieße die Ausführung abbzubrechen, falls die TCP Connection geschlossen wird.
Eine weitere Message für den Abbruch zu spezifizieren würde es ermöglichen, TCP Connections aufrecht zu erhalten wenn ein RPC abgebrochen wird.
Dies widerspricht aber dem Request-Response Konzept.
RPC Implementierungen wie \cite{qooxdoo} setzen dieses Konzept um.
Bei der Evaluierung von RPC Technologien konnte keine gefunden werden, die den Fortschritt der Ausführung am Client zur Verfügung stellt.
Um dies mit RPC zu realisieren, muss es im Application Layer implementiert werden.

\sys{Message-Oriented Middleware} Systeme bieten mehr Freiheit in der Hinsicht, dass Message-Sequenzen beliebig modelliert werden können.
Ein Remote Call mit Progress und Cancel Funktionalität kann relativ simpel mit vier Messages  modelliert werden:
Request, Progress, Cancel und Response.
Bei dieser Vorgehensweise muss jede der erwähnten Messages eine ID enthalten, welche auf den Aufruf verweist.

Beide Middleware Konzepte benötigen für die Progress und Cancel Funkionalität Implementierungen im Application Layer - eine Aufgabe, die dem Anwender abgenommen werden könnte.




\subsection{Asynchrone API}
Message-Oriented Middleware Systeme verwenden meist asynchrone APIs, während Procedure Oriented und Object Oriented Systeme meist synchrone APIs verwenden.
Es ist aber auch möglich, asynchrone APIs für RPC zu verwenden wie Falkner \cite{falkner1999implementing} gezeigt hat.

\label{future}
Asynchrone APIs, die mit Callbacks arbeiten, führen zu schwer lesbaren Code (pyramid of doom).
Um dies zu vermeiden können Futures und/oder Promises verwendet werden \cite{baker1977incremental}.
Promises sind Objekte, die das Ergebnis eines Funktionsaufrufs repräsentieren.
Bei den meisten Programmiersprachen kann dies neben dem Returnwert auch eine Exception sein.
Verwendet man Promises, wird das Ergebnis nicht durch Schlüsselwörter wie return oder throw definiert, sondern an das Promise Objekt übergeben.
So ist es möglich, auch innerhalb von Callbacks oder nach dem Verlassen der Funktion ein Ergebnis zu definieren.

Futures ergänzen dieses Konzept an der aufrufenden Stelle.
Asynchrone Funktionen geben beim Aufruf ein Future Objekt zurück.
Dieses kann verwendet werden um auf die Terminierung zu warten, beziehungsweise einen Eventhandler zu registrieren.
Wenn die asynchrone Funktion das Ergebnis an die Promise übergibt, gibt es diese an die Future weiter, welche in den Zustand Terminated übergeht.
Ist die Future in diesem Zustand, kann der Returnwert von ihr abgelesen werden \cite{baker1977incremental}.




\subsection{Active Object Pattern aka Actor Pattern}
Bei der Konzeptionierung von Serversystemen muss ein Threading Model gewählt werden.
Der Einsatz von multithreaded Modellen für ein Scripting API ist zu vermeiden, da es hohe Anforderungen an den Anwender im Bereich der Synchronisierung stellt.

Eine Möglichkeit Semaphoren zu vermeiden ist der Active Object Pattern \cite{schmidt2013pattern}.
Dieser verbindet eine Queue, in der Actions liegen, mit einem Thread welcher die Actions abarbeitet.
Es muss lediglich das Einfügen und Entfernen aus der Queue als atomare Operation ausgeführt werden.




\section{Technologien}

\subsection{Cluster- Grid- und Cloudcomputing}
Cluster, Grid und Cloud Computing haben eine Gemeinsamkeit:
Sie dienen der Nutzung von verteilten Ressourcen.
Ihre Unterschiede und die Abgrenzung zueinander sind nicht einheitlich definiert.
Im Folgenden wird die Klassifizierung nach \cite{sadashiv2011cluster} verwendet und ein kurzer Überblick in Bezug auf örtliche Ausdehnung, Ressourcen Allocation, Taskgröße, Heterogenität, Skalierbarkeit und Task Deployment gegeben.

Cluster Computing Systeme sind homogene, für lokale Netzwerke mit hohem Durchsatz optimierte Systeme.
Die Ressourcenverteilung wird von zentraler Stelle aus organisiert und limitiert damit die Skalierbarkeit.
Die Taskgröße ist nicht limitiert, dadurch eignen sie sich für verteilte Batch Verarbeitung.
Häufig wird MPI, eine Message Passing Middleware, verwendet.
Das Deployment der damit implementierten Programme ist nicht standardisiert und verlangt vom User detailliertes Wissen über den Cluster.

Grid Computing Systeme sind heterogene, für Netzwerke mit globaler Ausdehnung und geringem Durchsatz optimierte Systeme \cite{foster2003grid, foster2006grid}.
Typischerweise werden dabei mehrere lokale Netzwerke zu einem Grid verbunden, wodurch eine hierarchische Struktur entsteht.
Die Ressourcen Verteilung folgt ebenfalls dieser hierarchische Struktur, wodurch eine hohe Skalierbarkeit gegeben ist.
Die Taskgrößen sind nicht limitiert, dadurch eignen sie sich für verteilte Batch Verarbeitung.
Es existieren Erweiterungen wie MPICH-G2, die automatisches Job Deployment unterstützen.

Cloud Computing Systeme sind heterogene Systeme mit globaler Ausdehnung.
Sie sind für kleine Taskgröße designed und skalieren in Bezug auf die Anzahl dieser Tasks, aber nicht in Bezug auf die Große eines Tasks.
Diese Eigenschaft wird durch die Platform as a Service Eigenschaft erreicht, die das Task Deployment soweit vereinfacht, dass es on demand und automatisch geschehen kann.





\subsection{Apache spark}
Apache Spark ist ein Cluster Computing System, das auf einer MOM aufbaut.
Ein zu verteilender Algorithmus kann innerhalb eines Scripts implementiert werden.
Apache Spark API basiert auf Resilient Distributed Datasets (RDD).
RDDs bieten Methoden wie map, reduce, groupBy, filter und viele mehr.

Bei dem Aufruf dieser Funktionen unterteilt Apache Spark das RDD in Partitionen, wobei jeweils eine Partition von einem Worker verarbeitet wird.
Die hier als Beispiele genannten Methoden sind Funktionen höherer Ordnung, machen also klar, das auch ausführbarer Code an die Worker übergeben werden muss, damit dieser seine Partition abarbeiten kann.
Dies kann als versteckter Deployment Schritt gesehen werden und ist auch mit ein Grund für die einfache Handhabung von Apache Spark.

Darüber hinaus wird der User von technischen Details wie Fehlerkorrekturen und dem Versenden von Netzwerk Messages entbunden.
Apache  Spark unterstützt mehrere Persistenz Technologien, darunter HDFS, Hadoop und Cassandra \cite{zaharia2012resilient}.
