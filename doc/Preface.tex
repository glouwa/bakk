\chapter{Abstract}

Häufig ist es notwendig, Algorithmen auf eine große Anzahl von \InputDateien{} anzuwenden.
Dabei ist es bei einer verteilten Ausführung nicht notwendig den Algorithmus zu parallelisieren.
Es ist ausreichend, den gegebenen Algorithmus auf mehreren Geräten mit unterschiedlichen Inputs zu starten.
Existierende Tools wie \MapReduce{} oder \ApacheSpark{} sind dafür ausgelegt Algorithmen zu parallelisieren, und dadurch komplexer als notwendig.
Diese Arbeit beschäftigt sich mit dem Design, der Implementierung und Evaluierung einer leichtgewichtigen JavaScript Middleware.
Die Eigenschaft “leichtgewichtig” bezieht sich auf den Sourcecode- und API-Umfang.
Tasks mit langen Laufzeiten erfordern aus Sicht der Usability die Möglichkeit abgebrochen zu werden, eine \ProgressAnzeige{} und Informationen über den Endzustand des ausgeführten Algorithmus.

Diese Arbeit analysiert zunächst bestehende und geeignete Konzepte und Technologien für eine solche Middleware und stellt danach eine Implementierung einer Middleware mit \JavaScript{}, \node{}, \Websockets{} und asynchronem API vor.
Da durchgehend ein asynchrones API verwendet wird, benötigen die auf der Middleware aufgebauten Anwendungen keine Synchronisierungsmechanismen wie zum Beispiel Semaphoren.
Die Skalierbarkeit dieses Prototypen wird in Kapitel \ref{K7} anhand eines \rgAlgorithmus{} und einer verteilten \PrimzahlenSuche{} analysiert und zeigt lineare Skalierbarkeit bis zu 16 Worker.
